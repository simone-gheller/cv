%!TEX TS-program = xelatex
\documentclass[a4paper, extended]{comcv}

\usepackage[english]{babel}

\title{Simone Gheller's CV}
\fullname{Simone}{Gheller}{}
\cvtitle{Software Engineer}
\email{simone.gheller1@gmail.com}
\github{https://github.com/simone-gheller}{GitHub}
\phonenumber{(+39) 348 268 1923}
\currentdate{May 2024}

\begin{document}
\section{Work Experience}
\combosection{SumoLogic}{Software Engineer}{Milano (Italy) Oct 2022 - Present}{
  \begin{tightlist}
    \item Maintained the core app code by performing bug-fixing and working on improvements (eg new APIs) with Django, and developed UT, IT and E2E tests using Django and NodeJS with Cypress
    \item Implemented the company SRE rules to new products by leveraging the existing KPI system and applying it to a set of monitors to track the app health.
    \item Developed CI/CD pipelines following a declarative approach using Jenkins with the DSL plugin, and implementing the IaC pattern using Git, Terraform and AWS. 
  \end{tightlist}
}

\combosection{Capgemini}{Release Manager}{Milano (Italy) May 2021 - Oct 2022}{
    \begin{tightlist}
        \item Organized release elements to match the delivery with the schedule by assuming the role of coordinator between the dev team and the functional team.
        \item Performed troubleshooting and 1st level support on a Kubernetes cluster with linux based containers in case of outages, using Splunk, Dynatrace.
    \end{tightlist}
}

\combosection{Capgemini}{Release Manager}{Milano (Italy) Nov 2020 - May 2021}{
    \begin{tightlist}
        \item Developed DevOps chains that include cross-platform (web and mobile) app deployments and Continuous Quality mechanisms, with Jenkins, SonarQube, NodeJS, Docker.
        \item Developed CI/CD chains to support the dev team with quicker releases using Azure cloud
    \end{tightlist}
}

\section{Projects}
\combosection{\href{https://github.com/riccardo-nannini/Santorini/}{Santorini - BSc Thesis}}{Java, Maven, Swing}{}{
  \begin{tightlist}
    \item Transposition of the board game “Santorini” implemented in Java as an online multiplayer video game, performed as a member of a development team of three people.
  \end{tightlist}
}
\vspace{\topsep}
\combosection{\href{https://github.com/simone-gheller/gtoking}{GTO King}}{Javascript, NodeJs, React, Express, REST Apis}{}{
  \begin{tightlist}
    \item Developed a web App that simulates some case scenarios of a poker game. The app is completly built with Javascript and is composed by an api backend and a frontend based on React. 
  \end{tightlist}
}
\vspace{\topsep}

\section{Skills}
\combosection{Programming Languages}{}{}{
  Python, Javascript (+NodeJS), Java, Scala, C, Bash
}
\vspace{\topsep}
\combosection{Operations}{}{}{
  Docker, Kubernetes, AWS, OpenAPI std, Jenkins, Git, Splunk, Sumo Logic
}
\vspace{\topsep}
\combosection{Web frameworks}{}{}{
  Django (+DRF), Spring, React
}
\vspace{\topsep}
\combosection{Databases}{}{}{
  Postgres (+Citus), Redis, MySql, Oracle
}
\vspace{\topsep}
\combosection{languages}{}{}{
  English - professional, Italian - Native
}
\vspace{\topsep}


\section{Education}
\combosection{Politecnico di Milano}{B.S.c in Computer Engineering}{2016-2020}{
  Algorithms and Data structures, Relational Databases, Web technologies, Software Engineering, OOP, AOP, Networks, SDN, Economics and Business Organization
  }

\vspace{\topsep}


\end{document}